\clearpage
\chapter{Introduction}
\label{chap:introduction}

As the robotics industry develops, it is becoming increasingly common to see robots used to solve both important, difficult and dangerous tasks. Dealing with real world problems is often the goal when inventors work towards new technologies. Some inventions may be used in a multitude of ways and further the development of other increasingly complex solutions, while others serve society in small but significant ways for decades to come.
As the technologies mature and become more widely available, new possibilities arise. Possibilities to use the technology for less serious tasks and instead focus on showing its capabilities in a fun and eye-catching way. This could be done by using it in an unconventional environment or by completing a task far outside the intend of the inventor.
This project will try to use technology in such a way. It will try to use methods originally developed for the industry to complete an objective it was not intended for.

\section{Project Goals}

Using a UR5 robot arm with an attached gripper, a camera, and software, the aim of this project is to pick up a ball and throw it at a dart board placed within view of the camera. For safety reasons the ball thrown is light weight, made of plastic and is about the size of a ping pong ball. To allow it to stick to the dart board, it is coated with Velcro, just as the dart board is. Success will be measured based on how far the average throw is from the bullseye.

\section{Problem Definition} \label{sec:ProblemDefinition}

\textbf{Make a UR5 robot arm throw an object at a designated target}
\begin{itemize}
	\item Identify the target point using machine vision
	\item Plan a trajectory for the object using physics and kinematics
	\item Control a UR5 and gripper based on modeled trajectory
\end{itemize}
The solution is expected to consistently hit within the visual bullseye of the Velcro dart board, as seen in \textcolor{red}{figure \ref{fig:dartboard}}. Specifically, the center of the ball should be within 2 cm of the center of the bullseye. This accuracy would also allow the solution to hit within the opening of a standard American 16 oz red solo cup with a ping pong ball. The accuracy is accepted to be consistent, if at least 90\% of throws hit within the accepted target area.
Additionally, the solution should require no human intervention once started, apart from placing the ball in a designated pickup area at the start of each throw.

\section{Constraints \& Limitations}

The project is subject to constraints originating from physical limitations, and limitations set internally. Safety configurations limits robot's motion by both joint limits and Cartesian boundaries, while the goal is limited to throwing in one general direction rather than 360 degrees around the robot.

As only one camera is available at each robot cell, depth perception is limited and thus all target are assumed to be located at table height. All software will be developed and tested in a Linux based environment, support for any other operating system will not be considered. 

\section{System Overview}

The system developed in this project integrates several key components that together enable the robot to perceive its environment, plan feasible trajectories, and execute accurate movements on a calibrated workspace. The workflow is structured around five core modules: machine vision, trajectory planning, kinematics, robot–table calibration, and robot control. These modules operate sequentially but remain closely interconnected to ensure reliable performance throughout the robotic task.
\newline
\newline
Machine Vision provides the system’s perception capabilities. A camera captures images of the table surface, which is first calibrated. The system detects a circle on the dartboard and then calculates the center point of that circle. This point is extracted and interpreted in the camera coordinate frame, then converted into world-frame coordinates and forwarded to the robot–table calibration module. This information forms the basis for determining the target positions that the robot must reach.
\newline
\newline
Trajectory Planning is responsible for generating a smooth and collision-free path from the robot’s current configuration to the target location. Using the transformed vision data together with the robot’s kinematic constraints, the planner computes either a sequence of intermediate waypoints or a continuous motion profile that ensures safe, efficient, and feasible movement.
\newline
\newline
Kinematics forms the mathematical layer that converts desired end-effector positions into corresponding joint configurations. Inverse kinematics determines the joint angles required to reach a specified target pose, while forward kinematics is used to verify the resulting motion. This ensures that all movements generated by the system are physically achievable by the robot.
\newline
\newline
Robot Control executes the planned trajectory by commanding the robot’s joints to follow the desired motion profile. The control system ensures stable tracking and compensates for minor deviations or disturbances during execution, enabling the robot to reach the detected target points with high precision.
\newline
\newline
Together, these components form a cohesive and modular system in which perception, calibration, planning, kinematic computation, and control are tightly integrated. This structure enables the robot to operate reliably within the table workspace and interact with its environment in a precise and consistent manner. An overview of the complete experimental setup is shown in Figure~\ref{fig:SystemThrow}.


%system billede her 
\begin{figure}[htbp]
	\centering
	\includegraphics[width=0.7\textwidth]{images/SystemThrow.png}
	\caption{System overview of the robotic throwing setup.}
	\label{fig:SystemThrow}
\end{figure}

