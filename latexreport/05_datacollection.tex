\chapter{Data Collection}
\label{chap:datacollection}

In order to iteratively improve the performance and understand the behavior of the robotic system, a robust data collection mechanism was implemented. This process is critical for validating the calculated path against the robot's actual motion, particularly during the high-speed throwing sequence.

For each calculated throw, a comma-separated values (CSV) file is generated. The file is initially populated with the crucial input parameters and pre-computed trajectory data, including:
\begin{itemize}
	\item Target and release points
	\item Ballistic parameters (e.g., pitch, yaw, release velocity)
	\item Status code
	\item The complete pre-computed sequences of joint positions ($\mathbf{q}$) and velocities ($\mathbf{\dot{q}}$)
\end{itemize}
During the execution of the movement, the actual joint positions and velocities achieved by the UR5 are requested from the robot controller using the RTDE Receive Interface. These real-time values are simultaneously logged to the same CSV file. This combined dataset allows for a direct comparative analysis between the modeled throw (Matlab output) and the executed throw (RTDE feedback). 
To physically see where the ball ended up, a Velcro target is used. This target is placed on a white background, to help the vision distinguish between it and the table easier, and a 3d printed set of rings were placed at the centre. These rings are needed because the Velcro of the target became less sticky, allowing bounces and more movement than we could measure with properly. The rings start at 120mm inner diameter and go all the way down to 45mm. These rings go down in diameter by 1 cm for each ring, the only exception being the “ideal target” ring. This ideal was decided in the beginning of the project, being a circle with a 74mm diameter. Any hit in this circle would make it within 2cm of the centre. This ring was made slightly thicker, making the step larger, but also more stable as to not move this ring. The way these rings are used is by adding them one by one into the previous, allowing us to make our “target” smaller and test for higher precision. 
\textcolor{red}{INSERT IMAGE OF RINGS HERE}