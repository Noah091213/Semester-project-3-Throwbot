\chapter{Results}
\label{chap:results}

Testing the system and its components can be done in many ways. Because so many parameters can be tweaked, it was important to test many of them to determine their individual effects on the system as a whole. The results from all tests will be discussed in chapter \ref{chap:discussion}.

Initial testing was done using simulated throws, by running the trajectory planner, including the ball ballistics and robot path planning, with a range of release point and targets. The very first test was done to evaluate the impact of using a weighted Jacobian to influence which joints were most active during the throw movement. Table \ref{tab:jacobian_w_vs_u} shows the results from 5600 simulated throws across a range of release points and targets, the distribution of which can be seen in figure \ref{fig:subfig:Short}. 

\begin{table}[h!]
	\centering
	\caption{Simulation Test: Weighted vs Unweighted Jacobian}
	\label{tab:jacobian_w_vs_u}
	\begin{tabular}{rp{6.5cm}ccl}
		\toprule
		\textbf{Status} & \textbf{Description} & \textbf{Unweighted} & \textbf{Weighted} & \\
		\midrule
		21 & IK release position off by $> 5$ mm & 1424 & 1424 & \\
		23 & Joint limits exceeded in lead-up & 827 & 848 & (+21) \\
		24 & TCP limits exceeded & 156 & 156 &  \\
		26 & Release position too close to target & 144 & 144 &  \\
		100000 & Perfect Hit ($<1$mm offset, clean) & 78 & 91 & (+13) \\
		100100 & $<1$mm Offset + Follow-through Clipped & 22 & 0 & (-22) \\
		$1\text{X}0000$ & 1-5mm Offset (clean) & 97 & 135 & (+38) \\
		$1\text{X}0100$ & 1-5mm Offset + Follow-through Clipped & 52 & 2 & (-50) \\
		\midrule
		\multicolumn{2}{l}{Targets hit (of 28 possible)} & 20 & 20 & \\
		\multicolumn{3}{l}{Settings: Acceleration $5\ rad/s^2$, Min. distance 20 cm} & & \\
		\bottomrule
	\end{tabular}
\end{table}

Using a weighted Jacobian, which sets the cost of using Wrist 2 at twice the amount of all other joints, impacts both the lead-up and follow-through sequence. During lead-up, it becomes more common that a joint limit is exceeded, likely because some joints have to move more to compensate for the minimized movement in Wrist 2. However, when a path is found, it is more likely to be clean, i.e. not experiencing any clipping in the follow-through sequence. Both tests were able to reach the same amount of targets, with the targets near the side of the table being the unreachable ones.

Further simulation tests were done to evaluate the impact of the model's acceleration. Since all robot path planning is calculated based on the model's constant acceleration, changing this one parameter will likely dictate which targets are reachable. Table \ref{tab:acceleration_test} shows the impact of 3 different accelerations when enforcing a minimum horizontal distance of 20cm between release and target point. Additionally, table \ref{tab:acceleration_test_35cm} shows testing with a minimum distance of 35cm. The distribution of release points and targets for these tests can be seen in figure \ref{fig:subfig:Short} and \ref{fig:subfig:Long} respectively.

\begin{table}[h!]
	\centering
	\caption{Simulation Test: Acceleration}
	\label{tab:acceleration_test}
	\begin{tabular}{rp{6.5cm}cclcl}
		\toprule
		\textbf{Status} & \textbf{Description} & \textbf{Acc. $4\ rad/s^2$} & \multicolumn{2}{c}{\textbf{Acc. $5\ rad/s^2$}} & \multicolumn{2}{c}{\textbf{Acc. $6\ rad/s^2$}} \\
		\midrule
		21 & IK release position off by $> 5$ mm & 1423 & 1424 & (+1) & 1424 &  \\
		23 & Joint limits exceeded in lead-up & 960 & 848 & (-112) & 741 & (-107) \\
		24 & TCP limits exceeded & 120 & 156 & (+36) & 182 & (+26) \\
		26 & Release position too close to target & 144 & 144 & & 144 & \\
		100000 & Perfect Hit ($<1$mm offset, clean) & 59 & 91 & (+32) & 127 & (+36) \\
		100100 & $<1$mm Offset + Follow-through Clipped & 0 & 0 & & 0 &  \\
		$1\text{X}0000$ & 1-5mm Offset (clean) & 92 & 135 & (+43) & 180 & (+45) \\
		$1\text{X}0100$ & 1-5mm Offset + Follow-through Clipped & 2 & 2 &  & 2 &  \\
		\midrule
		\multicolumn{2}{l}{Targets hit (of 28 possible)} & 17 & 20 & (+3) & 20 & \\
		\multicolumn{7}{l}{Settings: Min. distance 20 cm, Weighted Jacobian} \\
		\bottomrule
	\end{tabular}
\end{table}

\begin{table}[h!]
	\centering
	\caption{Simulation Test: Acceleration at Distance > 35cm}
	\label{tab:acceleration_test_35cm}
	\begin{tabular}{rp{6.5cm}cclcl} 
		\toprule
		\textbf{Status} & \textbf{Description} & \textbf{Acc. $5\ rad/s^2$} & \multicolumn{2}{c}{\textbf{Acc. $6\ rad/s^2$}} & \multicolumn{2}{c}{\textbf{Acc. $7\ rad/s^2$}} \\ 
		\midrule
		21 & IK release position off by $> 5$ mm & 652 & 652 & & 653 & (+1) \\
		23 & Joint limits exceeded in lead-up & 2052 & 1945 & (-107) & 1818 & (-127) \\
		24 & TCP limits exceeded & 40 & 95 & (+55) & 181 & (+86) \\
		26 & Release position too close to target & 0 & 0 & & 0 & \\
		100000 & Perfect Hit ($<1$mm offset, clean) & 31 & 58 & (+27) & 75 & (+17) \\
		100100 & $<1$mm Offset + Follow-through Clipped & 0 & 0 & & 0 & \\
		$1\text{X}0000$ & 1-5mm Offset (clean) & 25 & 49 & (+24) & 71 & (+22) \\
		$1\text{X}0100$ & 1-5mm Offset + Follow-through Clipped & 0 & 1 & (+1) & 2 & (+1) \\
		\midrule
		\multicolumn{2}{l}{Targets hit (of 35 possible)} & 11 & 11 & & 12 & (+1) \\ 
		\multicolumn{7}{l}{Settings: Min. distance 35 cm, Weighted Jacobian} \\
		\bottomrule
	\end{tabular}
\end{table}

\begin{figure}[h!]
	\centering
	\begin{subfigure}[]{0.25\textwidth}
		\centering
		\includegraphics[width=\textwidth]{images/TableOverview1.png}
		\caption{Initial Testing}
		\label{fig:subfig:Short}
	\end{subfigure}
	\hfill
	\begin{subfigure}[]{0.25\textwidth}
		\centering
		\includegraphics[width=\textwidth]{images/TableOverview2.png}
		\caption{Distance Testing}
		\label{fig:subfig:Long}
	\end{subfigure}
	\hfill
	\begin{subfigure}[]{0.25\textwidth}
		\centering
		\includegraphics[width=\textwidth]{images/TableOverview3.png}
		\caption{Combined Overview}
		\label{fig:subfig:Combined}
	\end{subfigure}
	\hfill
	\begin{subfigure}[]{0.18\textwidth}
		\centering
		\includegraphics[width=\textwidth]{images/TableOverview4.png}
		\label{fig:subfig:Description}
	\end{subfigure}
	\caption{Overhead Visualization of Experimental Workspace and Target Distribution}
	\label{fig:TableOverview}
\end{figure}


\newpage

The tests show similar results, where increasing the acceleration decreases the risk of hitting a joint limit. However, the increase in acceleration also causes more throws to exceed the TCP limits. Additionally, it is seen that more viable paths are found and that more targets can indeed be reached.

\newpage
\noindent
In addition to the simulated test throws, a series of real-world throws were also conducted. Since the model and robot can have different accelerations applied to them, the relationship between the two could also impact the robot's precision. Figure \ref{fig:MAE_Velocity} shows the Mean Absolute Error in $rad/s$ across different relative accelerations. The model uses a constant acceleration of $5 rad/s^2$ while the robots acceleration is varied relative the model's. 

\begin{figure}[h!]
	\centering
	\includegraphics[width=\textwidth]{images/MAE_Velocity.png}
	\caption{Mean Absolute Velocity Error by Joint in $rad/s$. Model Acceleration $5 \ rad/s^2$}
	\label{fig:MAE_Velocity}
\end{figure}

It is seen that a slight increase in the robots acceleration relative to the model's, generally decreases the absolute mean error. For all further tests, the robot's acceleration will be set to 25\% more than the model's.

Table \ref{tab:test_throw} shows the results of 100 throws across different target objects and release timings. The two targets used were the dart board and the plastic ring with a diameter similar to the bullseye of the dart board. The release offset acts as a lead time for the gripper command. If the ball must lose contact at time step $t$, an offset of 2 triggers the release command at $t - 2$. The results are segmented into misses, boundary hits and clean hits. A clean hit is defined such that the entire ball must make it inside the bullseye or ring. A boundary hit is counted when the ball is partially inside the bullseye or hits the ring itself. A miss is noted if the ball does not meet one of the two hit requirements.

\begin{table}[h!]
	\centering
	\caption{Physical Test: Accuracy}
	\label{tab:test_throw}
	\begin{tabular}{llcccc}
		\toprule
		\textbf{Target} & \textbf{Release Offset} & \textbf{Misses} & \textbf{Boundary Hits} & \textbf{Clean Hits}  & \textbf{Success Rate} \\
		\midrule
		Dartboard & 2 Time Steps, 16ms & 9 & 9 & 7 & 28\% \\
		Plastic Ring & 2 Time Steps, 16ms & 10 & 5 & 10 & 40\% \\
		Plastic Ring & 5 Time Steps, 40ms & 9 & 7 & 9 & 36\% \\
		Plastic Ring & 6 Time Steps, 48ms & 5 & 3 & 17 & 68\% \\
		\bottomrule
	\end{tabular}
\end{table}

The first three setups show relatively similar results, with only the last setup being significantly different. The setup using a 48ms release offset resulted in only 20\% misses compared to around 40\% for the other three. Additionally, the last test showed a considerable increase in clean hits, with only 15\% of total hits being boundary hits. 

\newpage
\noindent
With the data from both the model and robot, it is possible to compare their velocities over time during a throw. Figure \ref{fig:VelocityOverTime} shows the velocity of both robot and model for the 4 most active joints during a throwing sequence. 

\begin{figure}[h!]
	\centering
	\begin{subfigure}[]{0.49\textwidth}
		\centering
		\includegraphics[width=\textwidth]{images/VelocityOverTimeBase.png}
		\caption{Base}
		\label{fig:subfig:Base}
	\end{subfigure}
	\hfill
	\begin{subfigure}[]{0.49\textwidth}
		\centering
		\includegraphics[width=\textwidth]{images/VelocityOverTimeShoulder.png}
		\caption{Shoulder}
		\label{fig:subfig:Shoulder}
	\end{subfigure}
	
	\vspace{3mm}
	
	\begin{subfigure}[]{0.49\textwidth}
		\centering
		\includegraphics[width=\textwidth]{images/VelocityOverTimeElbow.png}
		\caption{Elbow}
		\label{fig:subfig:Elbow}
	\end{subfigure}
	\hfill
	\begin{subfigure}[]{0.49\textwidth}
		\centering
		\includegraphics[width=\textwidth]{images/VelocityOverTimeWrist1.png}
		\caption{Wrist 1}
		\label{fig:subfig:Wrist1}
	\end{subfigure}
	\caption{Comparison of Velocity Over Time Between Model and Robot in $rad/s$}
	\label{fig:VelocityOverTime}
\end{figure}



\begin{comment}
	
	Tables for potential use
	
\begin{figure}[htbp]
	\centering
	\includegraphics[width=\textwidth]{images/VelocityProfile.png}
	\caption{Absolute Model Velocity Over Time in $rad/s$}
	\label{fig:VelocityProfile}
\end{figure}
	
\begin{table}[h]
	\centering
	\caption{Dartboard Tests}
	\label{tab:Dartboard_test}
	\begin{tabular}{clllll}
		\toprule
		\textbf{Target} & \textbf{Attempt 1} & \textbf{Attempt 2} & \textbf{Attempt 3} & \textbf{Attempt 4} & \textbf{Attempt 5} \\
		\midrule
		1 & Edge & Miss & Miss & Edge & Edge \\
		2 & Hit & Hit & Hit & Edge & Edge \\
		3 & Hit & Hit & Edge & Edge & Hit \\
		4 & Edge & Miss & Miss & Miss & Hit \\
		5 & Miss & Miss & Miss & Miss & Edge \\
		\midrule
		\multicolumn{6}{l}{Settings: Target Dartboard, Release offset 16ms} \\
		\bottomrule
	\end{tabular}
\end{table}

\begin{table}[h]
	\centering
	\caption{Plastic Ring Tests}
	\label{tab:Ring_test_16}
	\begin{tabular}{clllll}
		\toprule
		\textbf{Target} & \textbf{Attempt 1} & \textbf{Attempt 2} & \textbf{Attempt 3} & \textbf{Attempt 4} & \textbf{Attempt 5} \\
		\midrule
		1 & Miss & Edge & Edge & Edge & Miss \\
		2 & Hit & Hit & Miss & Hit & Hit \\
		3 & Hit & Hit & Hit & Hit & Hit \\
		4 & Miss & Miss & Edge & Miss & Hit \\
		5 & Edge & Miss & Miss & Miss & Miss \\
		\midrule
		\multicolumn{6}{l}{Settings: Target Plastic Ring, Release offset 16ms} \\
		\bottomrule
	\end{tabular}
\end{table}

\begin{table}[h]
	\centering
	\caption{Plastic Ring Tests}
	\label{tab:Ring_test_40}
	\begin{tabular}{clllll}
		\toprule
		\textbf{Target} & \textbf{Attempt 1} & \textbf{Attempt 2} & \textbf{Attempt 3} & \textbf{Attempt 4} & \textbf{Attempt 5} \\
		\midrule
		1 & Hit & Hit & Hit & Edge & Edge \\
		2 & Hit & Miss & Hit & Miss & Miss \\
		3 & Miss & Edge & Miss & Hit & Hit \\
		4 & Edge & Edge & Miss & Edge & Hit \\
		5 & Miss & Hit & Miss & Edge & Miss \\
		\midrule
		\multicolumn{6}{l}{Settings: Target Plastic Ring, Release offset 40ms} \\
		\bottomrule
	\end{tabular}
\end{table}

\begin{table}[h]
	\centering
	\caption{Plastic Ring Tests}
	\label{tab:Ring_test_48}
	\begin{tabular}{clllll}
		\toprule
		\textbf{Target} & \textbf{Attempt 1} & \textbf{Attempt 2} & \textbf{Attempt 3} & \textbf{Attempt 4} & \textbf{Attempt 5} \\
		\midrule
		1 & Hit & Hit & Hit & Miss & Hit \\
		2 & Hit & Hit & Hit & Hit & Hit \\
		3 & Hit & Hit & Hit & Edge & Hit \\
		4 & Miss & Hit & Hit & Edge & Hit \\
		5 & Hit & Miss & Edge & Miss & Miss \\
		\midrule
		\multicolumn{6}{l}{Settings: Target Plastic Ring, Release offset 48ms} \\
		\bottomrule
	\end{tabular}
\end{table}
\end{comment}

